\chapter{PSD的宇宙线地面标定}
\label{ch:cosmic_ray}
PSD整体装配完成后,我们在实验室对它的各项性能指标进行了测试。
特别地,我们利用宇宙线对PSD探测单元模块进行了细致的标定测试,得到了一系列的刻度参数结果。
这些结果是PSD得到的第一批刻度参数数据集,它们是PSD进行初步能量重建的基础。

我们专门研制并搭建了一套宇宙线地面标定系统对PSD整体进行标定测试。
本章将简单介绍该系统的基本组成以及它在PSD宇宙线标定中的应用,同时给出了PSD地面宇宙线标定的主要结果。

\section{宇宙线地面标定系统}
\label{sec:cosmic_ray:cm_system}
\subsection{简介}
\label{sec:cosmic_ray:introduction}
% 结构与组成,总装图
% 基本原理
宇宙线地面标定系统是专门为了PSD的地面标定而设计和建造的,主要提供了两个功能:1)模拟轨道真空环境;2)确定入射宇宙线径迹。
该系统的硬件组成如图\ref{fig:cosmic_ray:cm_system}所示,其主体为一个大型真空靶室,靶室大小正好可以容纳整个PSD探测器。
真空靶室的腔体上部和下部分别放置了一个多丝漂移室探测器(Multi-Wire Drift Chamber,简称MWDC)和一个塑料闪烁体触发板探测器。
上下两块大面积触发板用于标定入射宇宙线事例,而上下两个MWDC则用于测量入射宇宙线径迹。
根据得到宇宙线入射径迹,可以推出其在PSD上的击中位置;而MWDC和触发板的面积都能够覆盖PSD的有效面积,因此我们能够得到PSD所有位置处的MIP响应,从而实现PSD的精细标定。
\begin{figure}[htbp]
	\centering
	\includegraphics[width=0.65\textwidth]{chap/cosmic_ray/fig/cm_system.jpg}
	\caption{宇宙线地面标定系统的硬件组成}
	\label{fig:cosmic_ray:cm_system}
\end{figure}

按照功能分类,宇宙线地面标定系统可以被分为真空系统、径迹探测系统、触发系统以及数据获取系统。
下面,依次对各个分系统进行详细介绍。

\subsection{真空系统}
\label{sec:cosmic_ray:vacuum_system}
% 靶室大小,结构包括滑台,外观图
% 高压法兰,信号法兰图
% 真空泵,参数
真空系统主要由靶室、真空泵以及真空检测设备组成。
靶室腔体大小为$TODO$,可以容纳整个PSD探测器;靶室外部支架上设计有托盘,用于固定MWDC和塑闪触发板。
靶室内部还铺设了两根导轨,操作时,我们先将PSD安放在一个铝合金滑台上并准确定位,然后将滑台沿着导轨推入到靶室腔体中,最后将滑台用卡扣卡住以防其滑动。这样不仅方便了PSD的放置操作,而且能够准确定位PSD在靶室内部的位置,为后面的数据分析提供依据。
为了给靶室内部的PSD提供高压并将其信号引出,我们专门加工了PSD专用的高压转接法兰和信号转接法兰,如图TODO所示。

真空系统中的真空泵设备由机械泵和分子泵两级组成,以保证靶室内部可以达到较高的真空度。
相应的,真空检测设备也分为TODO。
由于靶室体积较大,我们配备了两套相同的真空泵设备以加快抽真空的速度。
实际使用中发现:当靶室内部放置了PSD探测器和相应的连接线缆后,TODO小时内可以将靶室内部真空抽取最低值多少。

\subsection{径迹探测系统}
\label{sec:cosmic_ray:tracking_system}
% MWDC工作原理
MWDC是气体探测器的一种,由于制造成本低廉、操作简单以及位置分辨较高,它是粒子物理实验中最常用的位置灵敏探测器之一。
MWDC一般由多个漂移单元组成,每个漂移单元都是MWDC进行位置测量的最小单位。
一个漂移单元由位于中心的一根阳极丝和周围的若干根阴极丝组成;工作时,阳极丝加正高压(或接地),而阴极丝接地(或加负高压);阴极丝也被称为场丝,因为它们的排布方式决定了漂移单眼内部的电场分布。
入射粒子穿过漂移单元时,在其中发生的物理过程可以叙述如下:
\begin{enumerate}
	\item 入射粒子使得单元内的气体分子电离,从而产生自由的电子离子对。
	\item 在电场作用下,电离出来的电子被加速并缓慢向阳极丝漂移,而离子则向阴极丝移动。
	\item 由于阳极丝附近电场强度非常大,自由电子漂移到这个区域以后迅速被加速到足够高的能量并能够引发新的电离。这个过程不断地反复地进行,从而产生大量电离电子,这个现象被称为雪崩效应。
	\item 雪崩效应产生的自由电子继续向阳极丝移动,由于距离阳极丝非常近,它们立即被阳极丝收集。
	\item 与此同时,原初电离以及次级电离产生的正离子继续向阴极丝移动,由于离子的质量大,它们的漂移速度很慢,而且也不会在阴极丝附近产生雪崩效应。
\end{enumerate}
% 1)入射粒子使得单元内的气体分子电离,从而产生自由的电子离子对;2)在电场作用下,电离出来的电子被加速并缓慢向阳极丝漂移,而离子则向阴极丝移动;3)由于阳极丝附近电场强度非常大,自由电子漂移到这个区域以后迅速被加速到足够高的能量并能够引发新的电离,这个过程不断地反复地进行,从而产生大量电离电子,这个现象叫作雪崩效应;4)雪崩效应产生的自由电子继续向阳极丝移动,由于距离阳极丝非常近,它们立即被阳极丝收集;5)与此同时,原初电离以及次级电离产生的正离子继续向阴极丝移动,由于离子的质量大,它们的漂移速度很慢,而且也不会在阴极丝附近产生雪崩效应。
上述各个步骤中,只有大量雪崩电子向阳极丝移动并被迅速吸收这个过程能够使得阳极丝上的电荷分布发生变化并感应出幅度可探测的电流脉冲信号。
MWDC探测到的也就是这个阳极丝上的感应电流信号,该信号产生时刻与入射粒子穿过漂移单元的时刻的时间差就是原初电离电子的漂移时间。
由于漂移距离与漂移时间直接关联,通过测量漂移时间可以反推出原初电离位置相对于阳极丝的位置,这就是MWDC进行位置测量的基本原理。

% 基本结构:面积,丝参数,精度。
% MWDC结构示意图(丝距离,丝面排布)
% MWDC实物图
径迹探测系统由两个完全一样的平面MWDC探测器组成。
每个MWDC探测器由X,Y,U三个方向的漂移单元平面阵列组成,其有效探测面积为$840mm \times 840mm$。
X面与Y面成\SI{90}{\degree}排布,它们内部都有80个漂移单元;U面与X面成\SI{30}{\degree}排布,它的内部有106个漂移单元,如图TODO所示。
漂移单元平面阵列是由两个阴极面和一个阳极面构成的,面间距为\SI{5}{mm}。
其中,阴极面由直径为\SI{100}{\micro\meter}的铍铜丝紧密排列组成,而阳极面由直径为\SI{100}{\micro\meter}的铍铜丝和直径为\SI{20}{\micro\meter}的镀金钨丝间隔排列组成,如图TODO所示。
工作时,所有的铍铜丝加负高压,它们是漂移单元的阴极丝;所有的镀金钨丝接地,它们是漂移单元的阳极丝。
因此,根据图\ref{}给出的间距,可以得到MWDC的漂移单元是一个截面为$10mm\times 10.5mm$的长方体。
MWDC探测器的工作气体为$\SI{20}{\percent}CO_2 + \SI{80}{\percent}Ar$,整个漂移腔体用Kapton膜密封,图TODO给出了一个MWDC探测器的实物图。

% MWDC的信号处理系统
MWDC的输出信号幅度较小,为了提高信噪比,我们使用一块基于SFE16芯片的前端电子学板对原始信号进行预处理,然后才将其输出到数据获取系统中进行时间测量(见\ref{sec:cosmic_ray:daq_system}节)。
SFE16是一款专门用于气体漂移室前端信号处理的低噪声ASIC芯片。
它将16个测量通道集成在一块芯片上,每个通道都具有信号放大和甄别的功能,因此输出信号可以直接用于时间测量,并适合长距离传输。
关于SFE16的详细信息可以参考文献\cite{sfe16}。

\subsection{触发系统}
\label{sec:cosmic_ray:triggering_system}
% 触发板结构,面积,厚度,四角读出,实物图?
触发系统由两块完全一样的塑料闪烁体触发板探测器组成。
触发板探测器使用\SI{30}{mm}厚的塑料闪烁体EJ-200\cite{ej-200}作为探测介质,探测器形状为正方形,有效面积为$825mm \times 825mm$。
塑闪板表面抛光,并在内层包裹Tyvek纸以提高反射效率,而在外层包裹黑胶带以屏蔽外界的光干扰。
最后,塑闪板的四个角各耦合了一支光电倍增管,用于闪烁光信号读出,如图TODO所示。
使用的光电倍增管型号为Hamamatsu公司的R7724\cite{r7724}。

% 触发板的信号处理系统
触发系统具有两个功能:1)标识入射宇宙线,为数据获取系统提供触发信号;2)用于宇宙线击中时刻的测量,为MWDC漂移时间的计算提供时间零点。
上下两块触发板探测器共有8路信号输出,相应地,它们经过甄别后可以得到8路时间信号。
每一路时间信号都扇出两路,一路信号参加8路时间信号的逻辑与运算;另一路信号用于时间测量。
上述两个功能都在同一块信号处理板中实现,详细内容参看\ref{sec:cosmic_ray:daq_system}节。
我们认为真实的宇宙线事例应该使得上下两块触发板都点火,因此8路信号相与得到的输出信号将作为触发信号输入到数据获取系统中以记录该事件。
而8个角上得到的时间测量值将被用于漂移时间零点的计算。

% 响应均匀性,击中位置图或者唐述文文章中的图
由于触发板的面积较大,宇宙线击中板上不同位置处产生的闪烁光传输到塑闪板上同一支PMT的时间是有差异的,我们把这个现象称为时间测量不均匀性,如图\ref{fig:cosmic_ray:tof_timeVSposition}所示。
\begin{figure}[htbp]
	\centering
	\includegraphics[width=0.65\textwidth]{chap/cosmic_ray/fig/tof_timeVSposition.png}
	\caption{触发板的时间测量不均匀性,引自\cite{tang_large_2015}。其中,纵轴是某个角得到的击中时刻,而横轴是击中位置距离这个角的直线距离。拟合直线斜率的倒数就是闪烁光在塑闪板中的有效传播速度。}
	\label{fig:cosmic_ray:tof_timeVSposition}
\end{figure}
这意味着,我们不能简单地选择一支PMT的时间信号作为时间零点,因为这会带来较大的时间晃动,影响漂移时间的测量精度。
为了消除时间测量不均匀性,必须综合板上四个角的时间测量信息,以使得时间零点与击中位置无关。
实际中,我们使用下列形式\cite{annand_large_1987}得到时间零点$T_{eff}$:
\begin{equation}
	T_{eff} = \frac{1}{4}\sum^4_{i=1}T_i - \frac{v_{eff}}{8d}[(T_3-T_1)^2+(T_4-T_2)^2]
	\label{eq:cosmic_ray:tof_time}
\end{equation}
其中,$v_{eff}$是闪烁光在塑闪板中有效传输速度,$d$是塑闪板的对角线长度,$T_1\sim T_4$是塑闪板四个角的时间测量值,并且$T_1和T_3$是对角,而$T_2和T_4$也是对角。
测试发现,使用公式\ref{eq:cosmic_ray:tof_time}得到的塑闪板时间分辨率可以达到\SI{350}{\pico\second}\cite{tang_large_2015}。

\subsection{数据获取系统}
\label{sec:cosmic_ray:daq_system}
PXI机箱图
与PSD系统的集成,触发Veto板。
系统整体图

% \section{多丝漂移室的宇宙线刻度}
% \subsection{漂移时间零点的确定}
% \subsection{从漂移时间谱提取s-t关系:Integration Method}
% \subsection{迭代修正s-t关系}
% \subsection{多丝漂移室的性能参数}
\section{PSD的宇宙线标定}

\section{标定结果}
\subsection{基线噪声}
加高压与未加高压的结果对比。
基线中心值。
电子学刻度?

\subsection{MIP响应}
中心MIP拟合示例图
中心值一致性
能量分辨率分布

\subsection{光衰减效应}

\subsection{Dy58比值}
拟合示例
分布图,与LED测试值对比
动态范围计算与分布

\subsection{探测效率}
探测效率判据
事例筛选
结果分布

\subsection{位置分辨}
位置重建原理
结果

\section{PSD的物理量重建研究}
\subsection{能量重建}
\subsection{位置重建}