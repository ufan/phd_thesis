\chapter{PSD的宇宙线地面标定}
\label{ch:cosmicray_calibration}
PSD整体装配完成后,我们在实验室对它的各项性能指标进行了测试。
特别地,我们利用宇宙线对PSD探测单元模块进行了细致的标定测试,得到了一系列的刻度参数结果。
这些结果是PSD得到的第一批刻度参数数据集,它们是PSD进行初步能量重建的基础。

我们专门研制并搭建了一套宇宙线地面标定系统对PSD整体进行标定测试。
本章将简单介绍该系统的基本组成以及它在PSD宇宙线标定中的应用,同时给出了PSD地面宇宙线标定的主要结果。

\section{宇宙线地面标定系统}
\subsection{简介}
结构与组成,总装图
基本原理

\subsection{真空系统}
靶室大小,结构包括滑台,外观图
高压法兰,信号法兰图

真空泵,参数

\subsection{径迹探测系统}
MWDC工作原理

基本结构:面积,丝参数,精度。
MWDC结构示意图(丝距离,丝面排布)
MWDC实物图

MWDC的信号处理系统

\subsection{触发系统}
触发板结构,面积,厚度,四角读出,实物图?
响应均匀性,击中位置图或者唐述文文章中的图

触发板的信号处理系统

\subsection{数据获取系统}
PXI机箱图
与PSD系统的集成,触发Veto板。
系统整体图

% \section{多丝漂移室的宇宙线刻度}
% \subsection{漂移时间零点的确定}
% \subsection{从漂移时间谱提取s-t关系:Integration Method}
% \subsection{迭代修正s-t关系}
% \subsection{多丝漂移室的性能参数}

\section{PSD宇宙线标定结果}
\subsection{基线噪声}
加高压与未加高压的结果对比。
基线中心值。
电子学刻度?

\subsection{MIPs响应}
中心MIP拟合示例图
中心值一致性
能量分辨率分布

\subsection{光衰减效应}

\subsection{Dy58比值}
拟合示例
分布图,与LED测试值对比
动态范围计算与分布

\subsection{探测效率}
探测效率判据
事例筛选
结果分布

\subsection{位置分辨}
位置重建原理
结果

\section{PSD的物理量重建研究}
\subsection{能量重建}
\subsection{位置重建}