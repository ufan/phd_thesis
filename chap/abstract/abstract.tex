% vim:ts=4:sw=4
% Copyright (c) 2014 Casper Ti. Vector
% Public domain.

\begin{cabstract}
% 中文测试文字。
自从1933年暗物质的概念被首次提出到现在,大量的天文观测结果证实了暗物质的存在。
今天,暗物质假说已经被天文学家们普遍接受,并在现有的大爆炸宇宙标准模型中扮演重要角色。
由于在粒子物理的标准模型中找不到符合暗物质基本性质的稳定粒子,暗物质的组成成分和粒子属性一直困扰着人们,不断地激发人们对暗物质研究的兴趣。
许多超出标准模型的新物理理论都预言了可能的暗物质候选粒子,对暗物质的探测和研究能够检验这些新的理论模型,并有可能导致物理学产生新的革命。
近年来,国际上开展了大量的暗物质探测实验,它们使用不同的探测技术手段对暗物质问题进行研究,大大加深了人们对暗物质的认知。

暗物质粒子探测卫星(DArk Matter Particle Explorer,简称DAMPE)是中国科学空间先导专项的首批四颗科学卫星之一。
它基于空间间接探测的方法,通过测量暗物质粒子湮灭或衰变产生的$TeV$量级的高能伽马射线和高能电子能谱来探测和研究暗物质粒子。
DAMPE于2015年12月17日发射升空并成功进入预定轨道,是目前国际上同类探测器中观测能段范围最宽、能量分辨率最优的空间暗物质粒子探测器。

塑闪阵列探测器(Plastic Scintillator Detector,简称PSD)是DAMPE粒子探测器的关键子探测之一,它主要有两个功能:1)协助DAMPE的BGO量能器进行$e/\gamma$鉴别;2)进行电荷测量,鉴别$Z=1\sim 20$的入射重离子种类。
本论文的主要内容围绕着PSD的设计和研制过程展开,主要有以下几个部分组成:
\begin{itemize}
  \item PSD通过测量入射粒子沉积能量的大小实现基本功能,而空间应用环境对PSD的整体设计提出了特别的要求,本文第\ref{ch:description}章对PSD的工作原理和设计进行了简单介绍。
  \item PSD采用模块化设计,由82个探测单元模块组成X、Y相互垂直的两个塑闪阵列平面。为了满足PSD的功能需求,我们估算出每个探测单元模块都需要覆盖\SI{0.1}{MIPs}到\SI{1400}{MIPs}的动态范围区间。为此,我们设计和实现了光电倍增管双打拿极引出的读出方案,并进行了详尽的宇宙线测试和重离子束流测试以验证该设计的合理性。这些工作是第\ref{ch:large_dynmaicrange}章的主要内容。
  \item 光电倍增管是影响PSD探测单元模块性能关键器件,为了得到最佳的探测器性能,往往需要对PSD的光电倍增管进行详尽地测试。为此,我们专门设计了一套批量PMT测试平台,并用该平台得到了PSD所有候选光电倍增管的增益曲线和Dy58比值增益曲线。这些工作将在第\ref{ch:pmt_test}章进行介绍。
  \item 空间实验的特殊性要求探测器各组件具有更高的稳定性和可靠性。第\ref{ch:construction}章介绍了PSD的建造过程,主要包括PMT组件和塑闪单元条组件的测试、筛选、生产以及质量控制,以及PSD的整体装配.
  \item PSD的建造完成后,需要进行完整细致的测试以得到其性能参数。我们专门设计和搭建了一套地面宇宙线标定测试平台,并使用该平台对PSD进行了长时间的宇宙线标定。第\ref{ch:cosmic_ray}章对该平台进行了介绍,同时给出了PSD的宇宙线标定的初步结果,包括基线噪声,MIP响应,能量分辨率,衰减曲线,探测效率以及位置分辨能力。
\end{itemize}


\end{cabstract}

% \cleardoublepage
\begin{eabstract}
Since first proposed by F. Zwicky in 1933, the existence of dark matter has been confirmed by a large amount of astronomical observations.
Today, the dark matter hypothesis has been generally accepted by most of the astronomical community and plays a central role in the standard model of Big Bang cosmology.
As the Standard Model of particle physics doesn't a viable dark matter candidate which posses all the basic properties of dark matter, the composition of dark matter and its particle attribution have puzzled the scientists for a long time and continue to stimulate the interest in dark matter research.
On the other hand, many new theories beyond the Standard Model have predicted new particles that turn out to be excellent dark matter candidates.
The detection and study of dark matter could be used to verify these new theory, and may even start a new revolution in physics.
In recent years, many experiments, using different kinds of detection technologies, have been carried out to study the dark matter problem.
All these efforts will help the scientists get a deeper understanding of dark matter.

The DArk Matter Particle Explorer (DAMPE) ,
\end{eabstract}

